\documentclass[conference]{IEEEtran}

\newcommand{\thetitle}{Non-Parametric Discrete Mixture Model Recovery via Nonnegative Matrix Factorization}

\input{packages}
\input{definitions}

\title{\vspace{-0.25em}\thetitle}
\author{
{\large{Stefan~Karpinski, John~R.~Gilbert, Elizabeth~M.~Belding}} \vspace{0.25em}\\
Department of Computer Science \\
University of California, Santa Barbara \vspace{0.35em}\\
\textit{\{sgk,gilbert,ebelding\}@cs.ucsb.edu}
}

\graphicspath{{graphics/}}

\begin{document}
\maketitle

Mixture modeling expresses probability densities as convex combinations of constituent probability distributions:
\begin{align}\eqn{mixture-model}
  q_i(x) = \sum_{j=1}^r w_{ij} p_j(x),
\end{align}
The $q_i$ and $p_j$ are density functions, and $w_{ij}$ are non-negative weights, summing to unity for each $i$.
In classical mixture modeling, the constituent density functions, $p_j$, are assumed to be from a known class of parametric distributions.
Various well established algorithms, such as expectation minimization (\caps{EM}), can optimally recover the weights, $w_{ij}$, given an observed sample of values from $q_i$.

In certain settings, however, mixture modeling is desirable, but the constituent distributions are neither known in advance, nor can they be assumed to be parametric.
In this work, we demonstrate how for discrete event spaces, nonnegative matrix factorization (\caps{NMF}) can be effectively used to simultaneously recover both weights and constituent distributions, given only a large collection of variably-sized samples from unknown mixtures.
% Our work addresses mixture models over discrete event spaces, but can be applied readily to continuous spaces, since continuous quantities can be discretized, and our techniques then applied to the resulting large discrete spaces. Since our approach makes no parametric assumptions, the results are no less valid.

For discrete event spaces, \Equation{mixture-model} is expressed succinctly as matrix multiplication.
Letting $Q_{ik} = q_i(k)$, $W_{ij} = w_{ij}$, and $P_{jk} = p_j(k)$ we have:
\begin{align}
  Q = WP.
\end{align}
% The matrices are all be row-stochastic.
The problem of inferring both the weights, $w_{ij}$, and constituent distributions, $p_j$, from a collection of mixtures, $q_i$, is equivalent to finding the factors $W$ and $P$ given $Q$.
The problem constrains all three matrices to be row-stochastic, meaning that all entries are nonnegative, with rows summing to unity.

\end{document}
