\documentclass[conference]{IEEEtran}

\newcommand{\thetitle}{Non-Parametric Discrete Mixture Model Recovery via Nonnegative Matrix Factorization}

%!TEX root = paper.tex

\usepackage[labelfont=bf,small]{caption}
\usepackage[font=small,labelfont=bf,position=top,nearskip=0em]{subfig}
\usepackage{cite,amsmath,amssymb,rotating,multirow,bigstrut,url,wrapfig,relsize,paralist,array,mathtools,units}
\usepackage[hyperfigures,bookmarks,bookmarksopen,bookmarksnumbered,colorlinks,linkcolor=black,citecolor=black,filecolor=blue,menucolor=black,pagecolor=blue,frenchlinks=true,pdftitle={\thetitle}]{hyperref}

%!TEX root = paper.tex

%% LABELING COMMANDS
\renewcommand{\sec}[1]{\label{sec:#1}}
\newcommand{\eqn}[1]{\label{eqn:#1}}
\newcommand{\fig}[1]{\label{fig:#1}}
\newcommand{\tab}[1]{\label{tab:#1}}
\newcommand{\thm}[1]{\label{thm:#1}}
\newcommand{\defn}[1]{\label{def:#1}}

%% REFERENCING COMMANDS
\newcommand{\Appendix}[1]{\hyperref[sec:#1]{Appendix~\ref*{sec:#1}}}
\newcommand{\Section}[1]{\hyperref[sec:#1]{Section~\ref*{sec:#1}}}
\newcommand{\Equation}[1]{\hyperref[eqn:#1]{Equation~\ref*{eqn:#1}}}
\newcommand{\Figure}[1]{\hyperref[fig:#1]{Figure~\ref*{fig:#1}}}
\newcommand{\Table}[1]{\hyperref[tab:#1]{Table~\ref*{tab:#1}}}
\newcommand{\Theorem}[1]{\hyperref[thm:#1]{Theorem~\ref*{thm:#1}}}
\newcommand{\Definition}[1]{\hyperref[def:#1]{Definition~\ref*{def:#1}}}

%% MATHEMATICAL NOTATIONS

% common algebraic domains
\newcommand{\N}{\mathbb{N}}
\newcommand{\Z}{\mathbb{Z}}
\newcommand{\Q}{\mathbb{Q}}
\newcommand{\R}{\mathbb{R}}

% standard operators & functors
\renewcommand{\Pr}{\mathrm{Pr}}
\newcommand{\Image}{\text{Im}}
\newcommand{\Kernel}{\text{Ker}}

% common constructs
\newcommand{\abs}[1]{{\left|#1\right|}}
\newcommand{\absx}[1]{{|#1|}}
\newcommand{\card}[1]{{\left|#1\right|}}
\newcommand{\cardx}[1]{{|#1|}}
\newcommand{\norm}[1]{{\lVert#1\rVert}}
\newcommand{\normx}[1]{{\Vert#1\Vert}}
\newcommand{\set}[1]{{\left\{#1\right\}}}
\newcommand{\setx}[1]{{\{#1\}}}
\newcommand{\parens}[1]{{\left(#1\right)}}
\newcommand{\parensx}[1]{{(#1)}}
\newcommand{\bracket}[1]{{\left[#1\right]}}
\newcommand{\bracketx}[1]{{[#1]}}
\newcommand{\seq}[1]{{\left<#1\right>}}
\newcommand{\seqx}[1]{{\lvert#1\rvert}}
\newcommand{\tuple}[1]{{\left<#1\right>}}
\newcommand{\tuplex}[1]{{\lvert#1\rvert}}
\newcommand{\floor}[1]{{\left\lfloor#1\right\rfloor}}
\newcommand{\floorx}[1]{{\lfloor#1\rfloor}}
\newcommand{\ceil}[1]{{\left\lceil#1\right\rceil}}
\newcommand{\ceilx}[1]{{\lceil#1\rceil}}
\newcommand{\round}[1]{{\left[#1\right]}}
\newcommand{\roundx}[1]{{[#1]}}
\newcommand{\fracx}[2]{{#1/#2}}
\newcommand{\fracp}[2]{{\left(\frac{#1}{#2}\right)}}
\newcommand{\fracpx}[2]{{(#1/#2)}}
\newcommand{\smallfrac}[2]{{\textstyle{\frac{#1}{#2}}}}

% standard notations
\newcommand{\trans}[1]{{#1}^T}
\newcommand{\inner}[2]{{#1}\trans{#2}}
\newcommand{\cross}{\times}
\newcommand{\tensor}{\otimes}
\newcommand{\directsum}{\oplus}
\newcommand{\iso}{\cong}
\newcommand{\union}{\cup}
\newcommand{\inter}{\cap}
\newcommand{\Union}{\bigcup}
\newcommand{\Inter}{\bigcap}
\newcommand{\conj}{\wedge}
\newcommand{\disj}{\vee}
\newcommand{\Conj}{\bigwedge}
\newcommand{\Disj}{\bigvee}
\newcommand{\defeq}{=}
\renewcommand{\emptyset}{\varnothing}
\renewcommand{\setminus}{\,\raisebox{1pt}{$\smallsetminus$}\,}
\newcommand{\eldiv}{\,./\,}
\newcommand{\diag}{\text{diag}}
\newcommand{\rStoch}{\text{rs}}

%% FORMATTING BEHAVIORS
\newcommand{\caps}[1]{{\smaller{#1}}}
\newcommand{\latin}[1]{\textit{#1}}
\newcommand{\defterm}[1]{\textit{#1}}
\newcommand{\newfootnote}[2]{\newcommand{#1}{\footnote{#2} }}
\renewcommand{\bullet}{\raisebox{2pt}{$\centerdot$}}
\renewcommand{\arraystretch}{1.3}


\title{\vspace{-0.25em}\thetitle}
\author{
{\large{Stefan~Karpinski, John~R.~Gilbert, Elizabeth~M.~Belding}} \vspace{0.25em}\\
Department of Computer Science \\
University of California, Santa Barbara \vspace{0.35em}\\
\textit{\{sgk,gilbert,ebelding\}@cs.ucsb.edu}
}

\graphicspath{{graphics/}}

\begin{document}
\maketitle

Mixture modeling expresses probability densities as convex combinations of constituent probability distributions:
\begin{align}\eqn{mixture-model}
  q_i(x) = \sum_{j=1}^r w_{ij} p_j(x),
\end{align}
Here $q_i$ and $p_j$ are density functions, and $w_{ij}$ are nonnegative weights, summing to unity for each $i$.
In classical mixture modeling, the constituent density functions, $p_j$, are assumed to be from some class of parametric distributions.
Various well established algorithms, using expectation minimization (\caps{EM}), can optimally recover the weights, $w_{ij}$, given an observed sample of values from $q_i$.

In certain settings, however, mixture modeling is desirable, but the constituent distributions are neither known in advance, nor can they be assumed to be parametric.
In this work, we demonstrate how, for discrete event spaces, nonnegative matrix factorization (\caps{NMF}) can be effectively used to simultaneously recover both weights and constituent distributions, given a large collection of variably-sized samples from mixtures.
% Our work addresses mixture models over discrete event spaces, but can be applied readily to continuous spaces, since continuous quantities can be discretized, and our techniques then applied to the resulting large discrete spaces. Since our approach makes no parametric assumptions, the results are no less valid.

For discrete event spaces, \Equation{mixture-model} is expressed succinctly as matrix multiplication.
Letting $Q_{ik} = q_i(k)$, $W_{ij} = w_{ij}$, and $P_{jk} = p_j(k)$ we have:
\begin{align}\eqn{mixture-model-matrix}
  Q = WP.
\end{align}
% The matrices are all be row-stochastic.
The problem of inferring both the weights, $w_{ij}$, and constituent distributions, $p_j$, from a collection of mixtures, $q_i$, is equivalent to finding the factors $W$ and $P$ given $Q$.
All three matrices are constrined to be row-stochastic, meaning that all entries are nonnegative, with rows summing to unity.

The problem of finding such a factorization is known as nonnegative matrix factorization.
Such factorizations are not unique, so perfect recovery of $W$ and $P$ cannot generally be achieved.
On the other hand, any exact factorization of $Q$, is an equally valid mixture model for the given data.
Since a variety of \caps{NMF} algorithms have been proposed, this problem is partially solved.
Several difficulties remain, however:
\begin{enumerate}
  \item $Q$ is not known exactly, only a finite sample for each distribution row of $Q$ is observed;
  \item The samples for the rows may not have uniform size;
  \item \caps{NMF} is known to be \caps{NP}-hard; thus, all efficient algorithms are merely heuristic, and may not yield adequate results.
\end{enumerate}
This list is not exhaustive, and we will address and discuss several further difficulties.

Our motivating application is mixture modeling for network flow traces, whose distributions of packet sizes and inter-packet intervals (as well as other parameters), can be effectively modeled as discrete mixture models, using \caps{NMF}~\cite{Karpinski:2008}.
In this setting, there are several particularly challenging aspects:
\begin{enumerate}
  \item The distribution of sample sizes is heavy-tailed, having few very large samples, and many very small samples, with most flows having only one to three packets.
  \item The constituent distributions are not uniformly represented: the most prevalent distribution is far more prevalent than the next, etc.; i.e. it is also heavy-tailed.
\end{enumerate}
We will demonstrate using simulated data why both of these properties make factor recovery particularly difficult.

% The first challenge is troubling because even if $Q$ can be factored exactly with small rank, if there are many small samples, the rank of the sample matrix, $S$, will be large, and its singular values will not 

\end{document}
