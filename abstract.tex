\documentclass[conference]{IEEEtran}

\newcommand{\thetitle}{Non-Parametric Discrete Mixture Model Recovery with Nonnegative Matrix Factorization}

\input{packages}
\input{definitions}

\title{\vspace{-0.25em}\thetitle}
\author{
{\large{Stefan~Karpinski, John~R.~Gilbert, Elizabeth~M.~Belding}} \vspace{0.25em}\\
Department of Computer Science \\
University of California, Santa Barbara \vspace{0.35em}\\
\textit{\{sgk,gilbert,ebelding\}@cs.ucsb.edu}
}

\graphicspath{{graphics/}}

\begin{document}
\maketitle

Mixture models express one probability density as a convex combination of other probability distributions:
\begin{align}
  q(x) = \sum w_i p_i(x).
\end{align}
The constituent density functions, $p_i$, may be taken from a finite set of known distributions, or more commonly a class of parametric distributions on the event space.
In such settings, there are well-established algorithms, such as expectation minimization (\caps{EM}), that can optimally recover the weights, $w_i$ for an observed sample of values from $q$.

In certain settings, however, even when a mixture model may be appropriate, the constituent distributions are neither known in advance, nor can they be assumed to be parametric.
In this work, we demonstrate how recent work on nonnegative matrix factorization can be used to effectively recover discrete mixture models without parametric assumptions on the constituent distributions.
The restriction that distributions be restricted to discrete event spaces is not overly onerous, since continuous variables can be quantized, and typically are quantized in practice in any case.



\end{document}

